\chapter{Introduction}
\label{chap:01}

\section{Problem statement}
\label{chap:01:01}
In the world of Big Data, data visualization tools and technologies are essential to analyze massive amounts of information and make data-driven decisions. Hence it is not a surprise that applications that tackle this need are in great demand. These applications use graphical representation of information and data by using visual elements like charts and graphs.

\section{Motivation}
\label{chap:01:02}
The idea started when I first came into contact with a major cloud service provider: Amazon Web Services also known as AWS.

AWS is a subsidiary of Amazon that offers reliable, scalable, and inexpensive cloud computing services. It provides on-demand cloud computing platforms to individuals, companies and governments, on a metered pay-as-you-go basis. In fact, these cloud computing web services provide a set of primitive, abstract technical infrastructure and distributed computing building blocks and tools. One of these services is Amazon Elastic Compute Cloud, which allows users to have at their disposal a virtual cluster of computers, available all the time, through the Internet. AWS's virtual machines emulate most of the attributes of a real computer including hardware (CPU(s) and GPU(s) for processing, local/RAM memory, hard-disk/SSD storage); a choice of operating systems; networking; and pre-loaded application software such as web servers, databases, etc. \cite{aws-overview}

At the beginning its API seemed hard to comprehend, but with time I came to realize how easy it was in fact. Hence my determination to work with as many of its services as possible. At first I started with simple services like S3, but then I got to understand and use services like ECS. The more services I used the more Stream Analysis began to expand.

On the other hand another motivation consisted in learning and understanding a popular Back-end framework which is: ASP .NET Core. The main motivation was that this particular framework uses the C Sharp programming language which is from the C++, Java family, but it is newer and it's syntax seems modern and in the same time easy to comprehend.

\section{Thesis structure}
\label{chap:01:03}

The thesis is structured as follows:

\begin{enumerate}
	\item \autoref{chap:02} - introduces the functional and nonfunctional aspects of the web application
	\item \autoref{chap:03} - presents the framework architecture
	\item \autoref{chap:04} - describes in-depth the programmatic side of the web application 
	\item \autoref{chap:05} - provides guidance on how to use the framework 
	\item \autoref{chap:06} - provides conclusion on the achieved results
\end{enumerate}
